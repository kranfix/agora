\documentclass[conference]{IEEEtran}

\usepackage[utf8]{inputenc}
\usepackage{subfiles}

% *** GRAPHICS RELATED PACKAGES ***
\ifCLASSINFOpdf
  \usepackage[pdftex]{graphicx}
  % declare the path(s) where your graphic files are
  % \graphicspath{{../pdf/}{../jpeg/}}
  % and their extensions so you won't have to specify these with
  % every instance of \includegraphics
  % \DeclareGraphicsExtensions{.pdf,.jpeg,.png}
\else
  % or other class option (dvipsone, dvipdf, if not using dvips). graphicx
  % will default to the driver specified in the system graphics.cfg if no
  % driver is specified.
  % \usepackage[dvips]{graphicx}
  % declare the path(s) where your graphic files are
  % \graphicspath{{../eps/}}
  % and their extensions so you won't have to specify these with
  % every instance of \includegraphics
  % \DeclareGraphicsExtensions{.eps}
\fi

% correct bad hyphenation here
\hyphenation{op-tical net-works semi-conduc-tor}


\begin{document}
%
% paper title
% can use linebreaks \\ within to get better formatting as desired
\title{Diseño e implementación de un brazo robótico con actividad reconfigurable por un usuario final sin intervención de personal calificado}


% author names and affiliations
% use a multiple column layout for up to three different
% affiliations
\author{
  \IEEEauthorblockN{Oliver Paucar}
  \IEEEauthorblockA{Universidad Nacional de Ingeniería\\
    Lima, PERU\\
    Email: oliwpaucar11@gmail.com}
  \and
  \IEEEauthorblockN{Gabriel García}
  \IEEEauthorblockA{Universidad Nacional de Ingeniería\\
    Lima, PERU\\
    Email: gabrconatabl@gmail.com}
  \and
  \IEEEauthorblockN{Frank A. Moreno Vera}
  \IEEEauthorblockA{Hack-Zeit Research\\
    Lima, PERU\\
    Email: frank.moreno@hackzeit.com}
}

% use for special paper notices
%\IEEEspecialpapernotice{(Invited Paper)}




% make the title area
\maketitle



% IEEEtran.cls defaults to using nonbold math in the Abstract.
% This preserves the distinction between vectors and scalars. However,
% if the conference you are submitting to favors bold math in the abstract,
% then you can use LaTeX's standard command \boldmath at the very start
% of the abstract to achieve this. Many IEEE journals/conferences frown on
% math in the abstract anyway.

% no keywords

\begin{abstract}
%\boldmath
The abstract goes here.
\end{abstract}


% For peer review papers, you can put extra information on the cover
% page as needed:
% \ifCLASSOPTIONpeerreview
% \begin{center} \bfseries EDICS Category: 3-BBND \end{center}
% \fi
%
% For peerreview papers, this IEEEtran command inserts a page break and
% creates the second title. It will be ignored for other modes.
\IEEEpeerreviewmaketitle



\section{Introducción}
% no \IEEEPARstart
Un robot es una entidad automatico inteligente

Los brazos robóticos, en el transcurso de los últimos años, ha sido muy 
utilizado en diferentes ámbitos, desde demostraciones aplicativas en 
instituciones educativas hasta industrias de automatización, en este último,
los brazos poseen distintas | funcionalidades así como distintos tipos de 
programación para cada tipo de trabajo que se desee que este realice.

En la industria de la automatización, para poder cambiar la funcionalidad
de un brazo robótico se necesita de gente especializada, ergo el costo “del 
conocimiento del ingeniero” es algo que la empresa tiene que cubrir, pero 
por qué no tener un sistema el cual pueda ser adaptable para cualquier 
necesidad, sin necesidad de una persona capacitada y de esta manera reducir 
costos. Con esto se tiene como finalidad realizar un sistema de tal manera 
que pueda ser utilizado por una persona que no estaría calificado en los 
estándares de automatización de nivel industrial.

El presente trabajo busca demostrar que es posible re configurar las 
funciones del brazo, sin la necesidad de crear un nuevo código.

% You must have at least 2 lines in the paragraph with the drop letter
% (should never be an issue)

\hfill 
 


% An example of a floating figure using the graphicx package.
% Note that \label must occur AFTER (or within) \caption.
% For figures, \caption should occur after the \includegraphics.
% Note that IEEEtran v1.7 and later has special internal code that
% is designed to preserve the operation of \label within \caption
% even when the captionsoff option is in effect. However, because
% of issues like this, it may be the safest practice to put all your
% \label just after \caption rather than within \caption{}.
%
% Reminder: the "draftcls" or "draftclsnofoot", not "draft", class
% option should be used if it is desired that the figures are to be
% displayed while in draft mode.
%
%\begin{figure}[!t]
%\centering
%\includegraphics[width=2.5in]{myfigure}
% where an .eps filename suffix will be assumed under latex, 
% and a .pdf suffix will be assumed for pdflatex; or what has been declared
% via \DeclareGraphicsExtensions.
%\caption{Simulation Results}
%\label{fig_sim}
%\end{figure}

% Note that IEEE typically puts floats only at the top, even when this
% results in a large percentage of a column being occupied by floats.


% An example of a double column floating figure using two subfigures.
% (The subfig.sty package must be loaded for this to work.)
% The subfigure \label commands are set within each subfloat command, the
% \label for the overall figure must come after \caption.
% \hfil must be used as a separator to get equal spacing.
% The subfigure.sty package works much the same way, except \subfigure is
% used instead of \subfloat.
%
%\begin{figure*}[!t]
%\centerline{\subfloat[Case I]\includegraphics[width=2.5in]{subfigcase1}%
%\label{fig_first_case}}
%\hfil
%\subfloat[Case II]{\includegraphics[width=2.5in]{subfigcase2}%
%\label{fig_second_case}}}
%\caption{Simulation results}
%\label{fig_sim}
%\end{figure*}
%
% Note that often IEEE papers with subfigures do not employ subfigure
% captions (using the optional argument to \subfloat), but instead will
% reference/describe all of them (a), (b), etc., within the main caption.


% An example of a floating table. Note that, for IEEE style tables, the 
% \caption command should come BEFORE the table. Table text will default to
% \footnotesize as IEEE normally uses this smaller font for tables.
% The \label must come after \caption as always.
%
%\begin{table}[!t]
%% increase table row spacing, adjust to taste
%\renewcommand{\arraystretch}{1.3}
% if using array.sty, it might be a good idea to tweak the value of
% \extrarowheight as needed to properly center the text within the cells
%\caption{An Example of a Table}
%\label{table_example}
%\centering
%% Some packages, such as MDW tools, offer better commands for making tables
%% than the plain LaTeX2e tabular which is used here.
%\begin{tabular}{|c||c|}
%\hline
%One & Two\\
%\hline
%Three & Four\\
%\hline
%\end{tabular}
%\end{table}


% Note that IEEE does not put floats in the very first column - or typically
% anywhere on the first page for that matter. Also, in-text middle ("here")
% positioning is not used. Most IEEE journals/conferences use top floats
% exclusively. Note that, LaTeX2e, unlike IEEE journals/conferences, places
% footnotes above bottom floats. This can be corrected via the \fnbelowfloat
% command of the stfloats package.

\section{objetivos generales}
Los brazos robóticos ya son muy conocidos y utilizados pero el
valor agregado es la reconfiguracion/reprogramacion mediante un
brazo robótico a escala.

Indagar a nivel nacional la necesidad de las empresas, y presentarle
esta solución económica y reutilizable al ser de bajo costo y de alta 
factibilidad.
   
Básicamente lo que se plantea es un brazo multifuncional, de esta manera
no hay necesidad de desechar un brazo, de esta manera disminuyendo el 
impacto ambiental.

\section{análisis mecánico}
\subfile{estructura_mecanica}

\section{modos de operación}
\subfile{modos_de_operacion}
   
\section*{}
La automatización y toma de datos se realiza a través de un brazo a
escala, esto quiere decir que no habrá necesidad de interactuar con
el brazo robótico.
De la fig. ,en un principio,se encontrará al brazo en un estado “inicial”, 
al mover la posición del switch (A=0) transita al estado seguimiento, si se 
presiona el pulsador durante un corto tiempo, se estará en el estado grabar 
y se obtendrá los datos de ese instante, pero si se mantiene presionado el 
pulsador, entrará al estado grabar y tomará datos, luego estará a la espera 
de tomar datos que sea distinta a la anterior, si se suelta el pulsador se 
entrará nuevamente al estado seguimiento, volviendo el switch a su posición 
media(C=1, B=1), se estará en el estado inicial, cambiando el switch a la 
posición (B=0) entrará al estado automatizado, si se presiona por un 
instante el pulsador, el brazo se detendrá(STOP) y si se vuelve pulsar 
seguirá con el movimiento(PLAY), en el cual comenzará a reproducir el 
movimiento gracias a los datos tomados con ayuda del brazo a escala.

\section{Desarrollo y alcance futuro}
Como objetivo proximo
En el caso de la aplicacion de este proyecto son 
Se espera que este proyecte influencie a implementar otras ... y pueda
apoyar a la investifacion sobre


% conference papers do not normally have an appendix


% use section* for acknowledgement






% trigger a \newpage just before the given reference
% number - used to balance the columns on the last page
% adjust value as needed - may need to be readjusted if
% the document is modified later
%\IEEEtriggeratref{8}
% The "triggered" command can be changed if desired:
%\IEEEtriggercmd{\enlargethispage{-5in}}

% references section

% can use a bibliography generated by BibTeX as a .bbl file
% BibTeX documentation can be easily obtained at:
% http://www.ctan.org/tex-archive/biblio/bibtex/contrib/doc/
% The IEEEtran BibTeX style support page is at:
% http://www.michaelshell.org/tex/ieeetran/bibtex/
%\bibliographystyle{IEEEtran}
% argument is your BibTeX string definitions and bibliography database(s)
%\bibliography{IEEEabrv,../bib/paper}
%
% <OR> manually copy in the resultant .bbl file
% set second argument of \begin to the number of references
% (used to reserve space for the reference number labels box)
\begin{thebibliography}{1}

\bibitem{IEEEhowto:kopka}
H.~Kopka and P.~W. Daly, \emph{A Guide to \LaTeX}, 3rd~ed.\hskip 1em plus
  0.5em minus 0.4em\relax Harlow, England: Addison-Wesley, 1999.

\end{thebibliography}




% that's all folks
\end{document}
