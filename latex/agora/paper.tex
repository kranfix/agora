\documentclass[conference]{IEEEtran}

\usepackage[utf8]{inputenc}
\usepackage{subfiles}

% *** GRAPHICS RELATED PACKAGES ***
\ifCLASSINFOpdf
  \usepackage[pdftex]{graphicx}
  % declare the path(s) where your graphic files are
  % \graphicspath{{../pdf/}{../jpeg/}}
  % and their extensions so you won't have to specify these with
  % every instance of \includegraphics
  % \DeclareGraphicsExtensions{.pdf,.jpeg,.png}
\else
  % or other class option (dvipsone, dvipdf, if not using dvips). graphicx
  % will default to the driver specified in the system graphics.cfg if no
  % driver is specified.
  % \usepackage[dvips]{graphicx}
  % declare the path(s) where your graphic files are
  % \graphicspath{{../eps/}}
  % and their extensions so you won't have to specify these with
  % every instance of \includegraphics
  % \DeclareGraphicsExtensions{.eps}
\fi

% correct bad hyphenation here
\hyphenation{op-tical net-works semi-conduc-tor}


\begin{document}
%
% paper title
% can use linebreaks \\ within to get better formatting as desired
\title{Diseño e implementación de un brazo robótico con actividad reconfigurable por un usuario final sin intervención de personal calificado}


% author names and affiliations
% use a multiple column layout for up to three different
% affiliations
\author{
  \IEEEauthorblockN{Oliver Paucar}
  \IEEEauthorblockA{Universidad Nacional de Ingeniería\\
    Lima, PERU\\
    Email: oliwpaucar11@gmail.com}
  \and
  \IEEEauthorblockN{Gabriel García}
  \IEEEauthorblockA{Universidad Nacional de Ingeniería\\
    Lima, PERU\\
    Email: gabrconatabl@gmail.com}
  \and
  \IEEEauthorblockN{Frank A. Moreno Vera}
  \IEEEauthorblockA{Hack-Zeit Research\\
    Lima, PERU\\
    Email: frank.moreno@hackzeit.com}
}

% use for special paper notices
%\IEEEspecialpapernotice{(Invited Paper)}




% make the title area
\maketitle



% IEEEtran.cls defaults to using nonbold math in the Abstract.
% This preserves the distinction between vectors and scalars. However,
% if the conference you are submitting to favors bold math in the abstract,
% then you can use LaTeX's standard command \boldmath at the very start
% of the abstract to achieve this. Many IEEE journals/conferences frown on
% math in the abstract anyway.

% no keywords

\begin{abstract}
%\boldmath
The abstract goes here.
\end{abstract}


% For peer review papers, you can put extra information on the cover
% page as needed:
% \ifCLASSOPTIONpeerreview
% \begin{center} \bfseries EDICS Category: 3-BBND \end{center}
% \fi
%
% For peerreview papers, this IEEEtran command inserts a page break and
% creates the second title. It will be ignored for other modes.
\IEEEpeerreviewmaketitle


\section{Introducción}

\subfile{introduccion}

\section{objetivos generales}

\subfile{objetivos_generales}

\section{análisis mecánico}

\subfile{estructura_mecanica}

\section{modos de operación}

\subfile{modos_de_operacion}
   
%\section*{}
La automatización y toma de datos se realiza a través de un brazo a
escala, esto quiere decir que no habrá necesidad de interactuar con
el brazo robótico.
De la fig. ,en un principio,se encontrará al brazo en un estado “inicial”, 
al mover la posición del switch (A=0) transita al estado seguimiento, si se 
presiona el pulsador durante un corto tiempo, se estará en el estado grabar 
y se obtendrá los datos de ese instante, pero si se mantiene presionado el 
pulsador, entrará al estado grabar y tomará datos, luego estará a la espera 
de tomar datos que sea distinta a la anterior, si se suelta el pulsador se 
entrará nuevamente al estado seguimiento, volviendo el switch a su posición 
media(C=1, B=1), se estará en el estado inicial, cambiando el switch a la 
posición (B=0) entrará al estado automatizado, si se presiona por un 
instante el pulsador, el brazo se detendrá(STOP) y si se vuelve pulsar 
seguirá con el movimiento(PLAY), en el cual comenzará a reproducir el 
movimiento gracias a los datos tomados con ayuda del brazo a escala.

\section{Desarrollo y alcance futuro}

\subfile{des_y_alc_futuro}

\begin{thebibliography}{1}

\bibitem{IEEEhowto:kopka}
H.~Kopka and P.~W. Daly, \emph{A Guide to \LaTeX}, 3rd~ed.\hskip 1em plus
  0.5em minus 0.4em\relax Harlow, England: Addison-Wesley, 1999.

\end{thebibliography}

% that's all folks
\end{document}